\documentclass[a4paper,14pt]{article}
\usepackage[utf8]{inputenc}   % Encodage des caractères du fichier source
\usepackage[T1]{fontenc}      % Encodage des caractères pour les polices
\usepackage[french]{babel}    % Optionnel : support pour la typographie française

\usepackage{amsmath}
\usepackage{amssymb}
\usepackage{graphicx}
\usepackage{listings}
\usepackage{hyperref}
\usepackage{geometry}
\usepackage{xcolor}

\geometry{margin=1in}

\lstset{
    language=C++,
    basicstyle=\ttfamily\small,
    keywordstyle=\color{blue}\bfseries,
    stringstyle=\color{red},
    commentstyle=\color{green},
    numbers=left,
    numberstyle=\tiny\color{gray},
    stepnumber=1,
    numbersep=10pt,
    backgroundcolor=\color{white},
    tabsize=4,
    showspaces=false,
    showstringspaces=false,
    breaklines=true
}


\title{TD 6 : Définition de classes, surcharge d'opérateurs}
\author{}
\date{}

\begin{document}

% Ligne horizontale supérieure
\hrule height 1pt
\vspace{0.1cm}
% Titre du document
\noindent
\makebox[\textwidth]{{\small Univ. Côte d'Azur} \hfill \textbf{EPU-MAM4} }
\makebox[\textwidth]{{\small 2024-2025} \hfill \textbf{Programmation C++}}
\begin{center}
{\Large \bf Définition de classes, surcharge d'opérateurs}
\end{center}
\hrule height 1pt
\vspace{0.5cm}
% Ligne horizontale inférieure



\section*{Introduction}

Le but de ce projet est de construire un solveur de gradient conjugué (CG) générique en utilisant la métaprogrammation template en C++. Vous allez utiliser les implémentations existantes de matrices denses et creuses (CSR) pour développer un solveur capable de résoudre des systèmes linéaires $Ax = b$, où $A$ est représentée dans l'un des deux formats.

\section*{Objectifs}

L'objectif de ce projet est de :
\begin{itemize}
    \item Comprendre et implémenter la métaprogrammation template en C++.
    \item Généraliser les implémentations existantes de matrices denses et CSR pour les intégrer dans un solveur CG générique.
    \item Tester et valider l'efficacité du solveur sur différentes instances de matrices.
\end{itemize}

\section*{Implémentations Fournies}

Les implémentations des matrices denses et CSR vous sont fournies. Vous devez les généraliser à l'aide de templates pour construire un solveur CG capable de travailler avec ces deux types de matrices.

\subsection*{Classe \texttt{DenseMatrix}}

\begin{lstlisting}[language=C++]
template <typename T>
class DenseMatrix {
public:
    DenseMatrix(int rows, int cols)
        : rows_(rows), cols_(cols), data_(new T[rows * cols]()) {}

    ~DenseMatrix() {
        delete[] data_;
    }
    
    T operator()(int row, int col) const {
        assert(row >= 0 && row < rows_);
        assert(col >= 0 && col < cols_);
        return data_[row * cols_ + col];
    }

    T& operator()(int row, int col) {
        assert(row >= 0 && row < rows_);
        assert(col >= 0 && col < cols_);
        return data_[row * cols_ + col];
    }

    std::vector<T> operator*(const std::vector<T>& x) const {
        assert(cols_ == x.size());
        std::vector<T> result(rows_, 0.0);

        for (int i = 0; i < rows_; ++i) {
            for (int j = 0; j < cols_; ++j) {
                result[i] += (*this)(i, j) * x[j];
            }
        }

        return result;
    }

    int rows() const { return rows_; }
    int cols() const { return cols_; }

private:
    int rows_, cols_;
    T* data_;
};
\end{lstlisting}

\subsection*{Classe \texttt{CSRMatrix}}

\begin{lstlisting}[language=C++]
template <typename T>
class CSRMatrix {
public:
    CSRMatrix(int rows, int cols)
        : rows_(rows), cols_(cols) {
        row_ptrs_.resize(rows + 1, 0);
    }

    void add_value(int row, int col, T value) {
        assert(row >= 0 && row < rows_);
        assert(col >= 0 && col < cols_);
        values_.push_back(value);
        col_indices_.push_back(col);
        row_ptrs_[row + 1]++;
    }

    void finalize() {
        for (int i = 1; i <= rows_; ++i) {
            row_ptrs_[i] += row_ptrs_[i - 1];
        }
    }

    std::vector<T> operator*(const std::vector<T>& x) const {
        assert(cols_ == x.size());
        std::vector<T> result(rows_, 0.0);

        for (int i = 0; i < rows_; ++i) {
            for (int j = row_ptrs_[i]; j < row_ptrs_[i + 1]; ++j) {
                result[i] += values_[j] * x[col_indices_[j]];
            }
        }

        return result;
    }

    int rows() const { return rows_; }
    int cols() const { return cols_; }

private:
    int rows_, cols_;
    std::vector<T> values_;
    std::vector<int> col_indices_;
    std::vector<int> row_ptrs_;
};
\end{lstlisting}

\section*{Travail à Réaliser}

\subsection*{1. Généralisation avec les Templates}

Vous devez généraliser les deux classes \texttt{DenseMatrix} et \texttt{CSRMatrix} en utilisant une interface commune basée sur les templates. Pour cela, créez une classe abstraite \texttt{MatrixBase} qui définira l'interface commune aux deux types de matrices.

\begin{lstlisting}[language=C++]
template <typename T>
class MatrixBase {
public:
    virtual ~MatrixBase() = default;

    virtual std::vector<T> operator*(const std::vector<T>& x) const = 0;
    virtual int rows() const = 0;
    virtual int cols() const = 0;
};
\end{lstlisting}

Modifiez ensuite les classes \texttt{DenseMatrix} et \texttt{CSRMatrix} pour qu'elles héritent de cette interface commune \texttt{MatrixBase}.

\subsection*{2. Implémentation du Solveur CG Générique}

Implémentez un solveur CG générique en utilisant la métaprogrammation template. Ce solveur doit être capable de résoudre des systèmes linéaires pour n'importe quel type de matrice dérivé de \texttt{MatrixBase}.

\begin{lstlisting}[language=C++]
template <typename Matrix, typename T>
std::vector<T> conjugate_gradient(const Matrix& A, const std::vector<T>& b, T tol = 1e-10) {
    int n = A.rows();
    std::vector<T> x(n, 0.0);
    std::vector<T> r = b;  // r = b - Ax, avec x initialise a 0, donc r = b
    std::vector<T> p = r;
    T rs_old = dot(r, r);

    for (int i = 0; i < n; ++i) {
        std::vector<T> Ap = A * p;
        T alpha = rs_old / dot(p, Ap);

        for (int j = 0; j < n; ++j) {
            x[j] += alpha * p[j];
            r[j] -= alpha * Ap[j];
        }

        T rs_new = dot(r, r);
        if (std::sqrt(rs_new) < tol) {
            break;
        }

        for (int j = 0; j < n; ++j) {
            p[j] = r[j] + (rs_new / rs_old) * p[j];
        }
        rs_old = rs_new;
    }

    return x;
}

// Fonction pour calculer le produit scalaire de deux vecteurs
template <typename T>
T dot(const std::vector<T>& a, const std::vector<T>& b) {
    assert(a.size() == b.size());
    T result = 0.0;
    for (size_t i = 0; i < a.size(); ++i) {
        result += a[i] * b[i];
    }
    return result;
}
\end{lstlisting}

\subsection*{3. Tests et Validation}

Testez votre solveur avec les deux types de matrices fournis. Par exemple, vous pouvez résoudre le système suivant :

\[
\begin{pmatrix}
4 & 1 & 0 \\
1 & 3 & 1 \\
0 & 1 & 2 \\
\end{pmatrix}
\begin{pmatrix}
x_1 \\
x_2 \\
x_3 \\
\end{pmatrix}
=
\begin{pmatrix}
1 \\
2 \\
3 \\
\end{pmatrix}
\]

Comparez les performances et la précision des résultats obtenus pour les matrices denses et creuses.

\section*{Rendu}

Le projet doit être rendu sous forme de code source C++ compilable accompagné d'un rapport détaillant votre approche, les défis rencontrés, et une analyse des résultats obtenus.

\end{document}
